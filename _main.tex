% Options for packages loaded elsewhere
\PassOptionsToPackage{unicode}{hyperref}
\PassOptionsToPackage{hyphens}{url}
%
\documentclass[
]{book}
\usepackage{amsmath,amssymb}
\usepackage{iftex}
\ifPDFTeX
  \usepackage[T1]{fontenc}
  \usepackage[utf8]{inputenc}
  \usepackage{textcomp} % provide euro and other symbols
\else % if luatex or xetex
  \usepackage{unicode-math} % this also loads fontspec
  \defaultfontfeatures{Scale=MatchLowercase}
  \defaultfontfeatures[\rmfamily]{Ligatures=TeX,Scale=1}
\fi
\usepackage{lmodern}
\ifPDFTeX\else
  % xetex/luatex font selection
\fi
% Use upquote if available, for straight quotes in verbatim environments
\IfFileExists{upquote.sty}{\usepackage{upquote}}{}
\IfFileExists{microtype.sty}{% use microtype if available
  \usepackage[]{microtype}
  \UseMicrotypeSet[protrusion]{basicmath} % disable protrusion for tt fonts
}{}
\makeatletter
\@ifundefined{KOMAClassName}{% if non-KOMA class
  \IfFileExists{parskip.sty}{%
    \usepackage{parskip}
  }{% else
    \setlength{\parindent}{0pt}
    \setlength{\parskip}{6pt plus 2pt minus 1pt}}
}{% if KOMA class
  \KOMAoptions{parskip=half}}
\makeatother
\usepackage{xcolor}
\usepackage{color}
\usepackage{fancyvrb}
\newcommand{\VerbBar}{|}
\newcommand{\VERB}{\Verb[commandchars=\\\{\}]}
\DefineVerbatimEnvironment{Highlighting}{Verbatim}{commandchars=\\\{\}}
% Add ',fontsize=\small' for more characters per line
\usepackage{framed}
\definecolor{shadecolor}{RGB}{248,248,248}
\newenvironment{Shaded}{\begin{snugshade}}{\end{snugshade}}
\newcommand{\AlertTok}[1]{\textcolor[rgb]{0.94,0.16,0.16}{#1}}
\newcommand{\AnnotationTok}[1]{\textcolor[rgb]{0.56,0.35,0.01}{\textbf{\textit{#1}}}}
\newcommand{\AttributeTok}[1]{\textcolor[rgb]{0.13,0.29,0.53}{#1}}
\newcommand{\BaseNTok}[1]{\textcolor[rgb]{0.00,0.00,0.81}{#1}}
\newcommand{\BuiltInTok}[1]{#1}
\newcommand{\CharTok}[1]{\textcolor[rgb]{0.31,0.60,0.02}{#1}}
\newcommand{\CommentTok}[1]{\textcolor[rgb]{0.56,0.35,0.01}{\textit{#1}}}
\newcommand{\CommentVarTok}[1]{\textcolor[rgb]{0.56,0.35,0.01}{\textbf{\textit{#1}}}}
\newcommand{\ConstantTok}[1]{\textcolor[rgb]{0.56,0.35,0.01}{#1}}
\newcommand{\ControlFlowTok}[1]{\textcolor[rgb]{0.13,0.29,0.53}{\textbf{#1}}}
\newcommand{\DataTypeTok}[1]{\textcolor[rgb]{0.13,0.29,0.53}{#1}}
\newcommand{\DecValTok}[1]{\textcolor[rgb]{0.00,0.00,0.81}{#1}}
\newcommand{\DocumentationTok}[1]{\textcolor[rgb]{0.56,0.35,0.01}{\textbf{\textit{#1}}}}
\newcommand{\ErrorTok}[1]{\textcolor[rgb]{0.64,0.00,0.00}{\textbf{#1}}}
\newcommand{\ExtensionTok}[1]{#1}
\newcommand{\FloatTok}[1]{\textcolor[rgb]{0.00,0.00,0.81}{#1}}
\newcommand{\FunctionTok}[1]{\textcolor[rgb]{0.13,0.29,0.53}{\textbf{#1}}}
\newcommand{\ImportTok}[1]{#1}
\newcommand{\InformationTok}[1]{\textcolor[rgb]{0.56,0.35,0.01}{\textbf{\textit{#1}}}}
\newcommand{\KeywordTok}[1]{\textcolor[rgb]{0.13,0.29,0.53}{\textbf{#1}}}
\newcommand{\NormalTok}[1]{#1}
\newcommand{\OperatorTok}[1]{\textcolor[rgb]{0.81,0.36,0.00}{\textbf{#1}}}
\newcommand{\OtherTok}[1]{\textcolor[rgb]{0.56,0.35,0.01}{#1}}
\newcommand{\PreprocessorTok}[1]{\textcolor[rgb]{0.56,0.35,0.01}{\textit{#1}}}
\newcommand{\RegionMarkerTok}[1]{#1}
\newcommand{\SpecialCharTok}[1]{\textcolor[rgb]{0.81,0.36,0.00}{\textbf{#1}}}
\newcommand{\SpecialStringTok}[1]{\textcolor[rgb]{0.31,0.60,0.02}{#1}}
\newcommand{\StringTok}[1]{\textcolor[rgb]{0.31,0.60,0.02}{#1}}
\newcommand{\VariableTok}[1]{\textcolor[rgb]{0.00,0.00,0.00}{#1}}
\newcommand{\VerbatimStringTok}[1]{\textcolor[rgb]{0.31,0.60,0.02}{#1}}
\newcommand{\WarningTok}[1]{\textcolor[rgb]{0.56,0.35,0.01}{\textbf{\textit{#1}}}}
\usepackage{longtable,booktabs,array}
\usepackage{calc} % for calculating minipage widths
% Correct order of tables after \paragraph or \subparagraph
\usepackage{etoolbox}
\makeatletter
\patchcmd\longtable{\par}{\if@noskipsec\mbox{}\fi\par}{}{}
\makeatother
% Allow footnotes in longtable head/foot
\IfFileExists{footnotehyper.sty}{\usepackage{footnotehyper}}{\usepackage{footnote}}
\makesavenoteenv{longtable}
\usepackage{graphicx}
\makeatletter
\def\maxwidth{\ifdim\Gin@nat@width>\linewidth\linewidth\else\Gin@nat@width\fi}
\def\maxheight{\ifdim\Gin@nat@height>\textheight\textheight\else\Gin@nat@height\fi}
\makeatother
% Scale images if necessary, so that they will not overflow the page
% margins by default, and it is still possible to overwrite the defaults
% using explicit options in \includegraphics[width, height, ...]{}
\setkeys{Gin}{width=\maxwidth,height=\maxheight,keepaspectratio}
% Set default figure placement to htbp
\makeatletter
\def\fps@figure{htbp}
\makeatother
\setlength{\emergencystretch}{3em} % prevent overfull lines
\providecommand{\tightlist}{%
  \setlength{\itemsep}{0pt}\setlength{\parskip}{0pt}}
\setcounter{secnumdepth}{5}
\usepackage{booktabs}
\ifLuaTeX
  \usepackage{selnolig}  % disable illegal ligatures
\fi
\usepackage[]{natbib}
\bibliographystyle{apa}
\IfFileExists{bookmark.sty}{\usepackage{bookmark}}{\usepackage{hyperref}}
\IfFileExists{xurl.sty}{\usepackage{xurl}}{} % add URL line breaks if available
\urlstyle{same}
\hypersetup{
  pdftitle={ManyBabies 5 Lab Manual - DRAFT - DO NOT USE},
  pdfauthor={compiled by ManyBabies 5 Leads},
  hidelinks,
  pdfcreator={LaTeX via pandoc}}

\title{ManyBabies 5 Lab Manual - DRAFT - DO NOT USE}
\usepackage{etoolbox}
\makeatletter
\providecommand{\subtitle}[1]{% add subtitle to \maketitle
  \apptocmd{\@title}{\par {\large #1 \par}}{}{}
}
\makeatother
\subtitle{Primary Manual for Participating Laboratories}
\author{compiled by ManyBabies 5 Leads}
\date{Updated: 2024-04-26}

\usepackage{amsthm}
\newtheorem{theorem}{Theorem}[chapter]
\newtheorem{lemma}{Lemma}[chapter]
\newtheorem{corollary}{Corollary}[chapter]
\newtheorem{proposition}{Proposition}[chapter]
\newtheorem{conjecture}{Conjecture}[chapter]
\theoremstyle{definition}
\newtheorem{definition}{Definition}[chapter]
\theoremstyle{definition}
\newtheorem{example}{Example}[chapter]
\theoremstyle{definition}
\newtheorem{exercise}{Exercise}[chapter]
\theoremstyle{definition}
\newtheorem{hypothesis}{Hypothesis}[chapter]
\theoremstyle{remark}
\newtheorem*{remark}{Remark}
\newtheorem*{solution}{Solution}
\begin{document}
\maketitle

{
\setcounter{tocdepth}{1}
\tableofcontents
}
\hypertarget{overview}{%
\chapter*{Overview}\label{overview}}
\addcontentsline{toc}{chapter}{Overview}

\includegraphics[width=0.3\textwidth,height=\textheight]{images/mb-logo.png} \includegraphics[width=0.3\textwidth,height=\textheight]{images/mb5-logo.png}

Thank you for contributing to \href{https://manybabies.org/MB5/}{\textbf{ManyBabies 5}} (MB5), a project of \href{https://manybabies.org/}{\textbf{ManyBabies}}, a cross-lab effort to provide an empirical basis for discussions of replicability as well as cultural, developmental, and methodological variability in infant perception/cognition research. In this project, we are examining drivers of infants' familiarity vs.~novelty preference through a collaboratively-designed ``best test'' of a popular model of infants' visual preference for familiar and novel stimuli \citep{hunterames}. More details about the background, design and hypotheses can be found in the \href{https://osf.io/preprints/psyarxiv/ck3vd}{\textbf{Registered Report}}. Below we provide instructions on how to implement the experiment in your lab and report data back to the project as a whole.

\hypertarget{mb5-project-website-manybabies.orgmb5}{%
\paragraph*{\texorpdfstring{\textbf{MB5 Project website:} \url{manybabies.org/MB5}}{MB5 Project website: manybabies.org/MB5}}\label{mb5-project-website-manybabies.orgmb5}}
\addcontentsline{toc}{paragraph}{\textbf{MB5 Project website:} \url{manybabies.org/MB5}}

\hypertarget{manybabies-general-manual-link}{%
\paragraph*{\texorpdfstring{\textbf{ManyBabies General Manual:} \href{https://docs.google.com/document/d/1dZ3sF2UcxvpkfOfKSKFeObTMZRbpUYloMUiPYtZy0ng/edit?usp=sharing}{link}}{ManyBabies General Manual: link}}\label{manybabies-general-manual-link}}
\addcontentsline{toc}{paragraph}{\textbf{ManyBabies General Manual:} \href{https://docs.google.com/document/d/1dZ3sF2UcxvpkfOfKSKFeObTMZRbpUYloMUiPYtZy0ng/edit?usp=sharing}{link}}

\hypertarget{mb5-collaboration-agreement-link}{%
\paragraph*{\texorpdfstring{\textbf{MB5 Collaboration Agreement:} \href{https://docs.google.com/document/d/1vbTDmH6euda5pJN4uyds3zsnQ1DXrW9wpHogwC-5TSk/edit?usp=sharing}{link}}{MB5 Collaboration Agreement: link}}\label{mb5-collaboration-agreement-link}}
\addcontentsline{toc}{paragraph}{\textbf{MB5 Collaboration Agreement:} \href{https://docs.google.com/document/d/1vbTDmH6euda5pJN4uyds3zsnQ1DXrW9wpHogwC-5TSk/edit?usp=sharing}{link}}

\hypertarget{mb5-stage-1-registered-report-kosiezettersten2024}{%
\paragraph*{\texorpdfstring{\textbf{MB5 Stage 1 Registered Report:} \citep{kosiezettersten2024}}{MB5 Stage 1 Registered Report: {[}@kosiezettersten2024{]}}}\label{mb5-stage-1-registered-report-kosiezettersten2024}}
\addcontentsline{toc}{paragraph}{\textbf{MB5 Stage 1 Registered Report:} \citep{kosiezettersten2024}}

\hypertarget{mb5-contact-mb5manybabies.org}{%
\paragraph*{\texorpdfstring{\textbf{MB5 Contact:} \url{mb5@manybabies.org} }{MB5 Contact: mb5@manybabies.org }}\label{mb5-contact-mb5manybabies.org}}
\addcontentsline{toc}{paragraph}{\textbf{MB5 Contact:} \url{mb5@manybabies.org} }

\begin{center}\rule{0.5\linewidth}{0.5pt}\end{center}

\hypertarget{lab-checklist}{%
\section*{Lab Checklist}\label{lab-checklist}}
\addcontentsline{toc}{section}{Lab Checklist}

\begin{enumerate}
\def\labelenumi{\arabic{enumi}.}
\tightlist
\item
  If you are new to MB5, complete the \href{https://docs.google.com/forms/d/e/1FAIpQLSdJnP3KO_dCmj-jNPHs0XP2j3q66g1RI6L31dwhCzhwhoJeoA/viewform}{Initial Sign-Up Form}.
\item
  Read this manual start to finish.
\end{enumerate}

\hypertarget{before-you-begin-data-collection}{%
\subsubsection*{BEFORE you begin data collection:}\label{before-you-begin-data-collection}}
\addcontentsline{toc}{subsubsection}{BEFORE you begin data collection:}

\begin{enumerate}
\def\labelenumi{\arabic{enumi}.}
\setcounter{enumi}{2}
\tightlist
\item
  Please ensure that you have carefully read the \href{https://docs.google.com/document/d/1vbTDmH6euda5pJN4uyds3zsnQ1DXrW9wpHogwC-5TSk/edit?usp=sharing}{MB5 Collaboration Agreement} and all of the documentation from the \href{https://docs.google.com/document/d/1dZ3sF2UcxvpkfOfKSKFeObTMZRbpUYloMUiPYtZy0ng/edit?usp=sharing}{ManyBabies General Manual} regarding \href{https://docs.google.com/document/d/1dZ3sF2UcxvpkfOfKSKFeObTMZRbpUYloMUiPYtZy0ng/edit\#heading=h.22i70rxou3ha}{ethical research}, \href{https://docs.google.com/document/d/1dZ3sF2UcxvpkfOfKSKFeObTMZRbpUYloMUiPYtZy0ng/edit\#heading=h.9ty2g48mpe0t}{authorship}, \href{https://docs.google.com/document/d/1dZ3sF2UcxvpkfOfKSKFeObTMZRbpUYloMUiPYtZy0ng/edit\#heading=h.aunbjkpwxhf3}{data sharing}, and \href{https://docs.google.com/document/d/1dZ3sF2UcxvpkfOfKSKFeObTMZRbpUYloMUiPYtZy0ng/edit\#heading=h.6h67zsyeiveg}{data use}.
\item
  Set up your study in consultation with this document. Carefully record any needed deviations from the protocol. Decide on your planned sample size/stopping rule.
\item
  Complete the Laboratory Questionnaire {[}insert link{]}, submit Ethics approval and other documentation/materials.
\item
  Create and submit your walkthrough video.
\item
  Run pilot sample through \href{https://manybabies.org/validator/}{data validator}.
\item
  Send email to \url{mb5@manybabies.org} to let the leadership team know that you are ready for `greenlighting'.
\item
  Wait for your official ``greenlight'' from the leadership team to begin data collection.
\end{enumerate}

\hypertarget{data-collection}{%
\subsubsection*{Data collection:}\label{data-collection}}
\addcontentsline{toc}{subsubsection}{Data collection:}

\hypertarget{important-do-not-begin-data-collection-other-than-piloting-until-you-have-been-explicitly-and-individually-greenlighted-to-do-so.}{%
\paragraph*{IMPORTANT: DO NOT begin data collection (other than piloting) until you have been explicitly (and individually) ``greenlighted'' to do so.}\label{important-do-not-begin-data-collection-other-than-piloting-until-you-have-been-explicitly-and-individually-greenlighted-to-do-so.}}
\addcontentsline{toc}{paragraph}{IMPORTANT: DO NOT begin data collection (other than piloting) until you have been explicitly (and individually) ``greenlighted'' to do so.}

\begin{enumerate}
\def\labelenumi{\arabic{enumi}.}
\setcounter{enumi}{9}
\tightlist
\item
  Collect your data!
\end{enumerate}

\hypertarget{after-you-finish-data-collection}{%
\subsubsection*{AFTER you finish data collection:}\label{after-you-finish-data-collection}}
\addcontentsline{toc}{subsubsection}{AFTER you finish data collection:}

\begin{enumerate}
\def\labelenumi{\arabic{enumi}.}
\setcounter{enumi}{10}
\tightlist
\item
  Complete your participant and trial data files in consultation with the data reporting instructions {[}insert link{]}.
\item
  Submit your data using the \href{https://docs.google.com/forms/d/e/1FAIpQLSdFYk-gb4yjRYLjSTP1_BVaW-3vLkpJClLoY2BOGDGfIVE5ww/viewform?usp=sf_link}{MB5 data upload form}.
\end{enumerate}

\begin{center}\rule{0.5\linewidth}{0.5pt}\end{center}

\hypertarget{getting-started}{%
\chapter{Getting Started}\label{getting-started}}

\hypertarget{start-and-end-date}{%
\section{Start and End Date}\label{start-and-end-date}}

Data collection officially started XXX and will initially run until XXX. Labs may join the MB5 project any time during the data collection period, provided that set-up and a formal ``green light'' is obtained, and data collection can be completed before the end date. However, we understand that there may be disruptions to data collection for various reasons. If you are having trouble meeting this timeframe, please alert the leadership team to discuss possibilities.

\hypertarget{ethics-approval}{%
\section{Ethics approval}\label{ethics-approval}}

It is a good idea to get started on your ethics approvals as soon as possible. Approvals must be in place and a copy submitted to the Drive folder (see below) before you can obtain a green light to collect data. Contact us \href{mailto:mb5@manybabies.org}{\nolinkurl{mb5@manybabies.org}} if you need advice on obtaining ethics approval.

MB5 and other ManyBabies projects ask labs to collect a number of \href{ADD\%20LINK\%20TO\%20PARTICIPANTS\%20SECTION}{demographic background variables} to be made public along with the main data, e.g.~parental education, sex, etcetera. Not all of those variables are part of the main analysis. Even so, it is highly relevant to collect them for follow-up and secondary analyses, and for checking the representativeness of the sample. Please contact us \href{mailto:mb5@manybabies.org}{\nolinkurl{mb5@manybabies.org}} if your ethics board raises concerns about collecting these data.

\hypertarget{please-make-sure-you-have-permission-to-publicly-share-anonymized-raw-data-including-required-demographic-info-as-this-is-a-condition-of-participation.}{%
\paragraph*{Please make sure you have permission to publicly share anonymized raw data (including required demographic info), as this is a condition of participation.}\label{please-make-sure-you-have-permission-to-publicly-share-anonymized-raw-data-including-required-demographic-info-as-this-is-a-condition-of-participation.}}
\addcontentsline{toc}{paragraph}{Please make sure you have permission to publicly share anonymized raw data (including required demographic info), as this is a condition of participation.}

\hypertarget{video-sharing}{%
\subsection{Video sharing}\label{video-sharing}}

We strongly encourage labs (where possible) to store/share video recordings of their testing on \href{https://nyu.databrary.org/}{Databrary}, which is a secure site for this purpose. You will need to ensure that you have ethics approval in place to do this, and collect specific consent for this purpose from your participants. In addition, you will become a member of Databrary, which will require approval from your institution, so it's helpful to start this process early. You can \href{https://nyu.databrary.org/user/register?page=create}{begin the registration process for Databrary here}. We are aware that many laboratories, particularly in the European Union, may not be allowed to use Databrary. In these circumstances we encourage labs to use alternative methods for sharing their videos where possible.

\hypertarget{upload-a-copy-of-your-ethics-approval-documentation}{%
\subsection{Upload a copy of your ethics approval documentation}\label{upload-a-copy-of-your-ethics-approval-documentation}}

Ethics documentation (e.g., IRB approval forms) should be submitted prior to data collection. Ethics forms should be uploaded using the \href{https://docs.google.com/forms/d/e/1FAIpQLScTTmcQl1P1F4UWe95Jo7u5bken40AyAefXCYUJ9iYbnWaG8Q/viewform?usp=sf_link}{MB5 Documentation Upload Form}. \textbf{Please ensure that all materials uploaded for ethics are clearly labeled in the filename with your LabID (e.g.~\emph{babylabPrinceton\_ethics.pdf}).} Check \href{https://manybabies.org/labids/}{here} for a complete list of LabIDs to confirm yours before uploading. In some cases, labs will already have approval for MB5 under an ``umbrella protocol'' (i.e., a protocol that covers multiple studies in one lab). In these cases, the ethics form should still be uploaded as described here.

\hypertarget{participants-and-recruitment}{%
\section{Participants and Recruitment}\label{participants-and-recruitment}}

\hypertarget{age-and-numbers}{%
\subsection{Age and numbers}\label{age-and-numbers}}

The minimum expected contribution is a full sample of 32 babies (preferred, if possible) or a half sample of 16 babies between 3 months, 0 days and 15 months, 0 days old.

\begin{itemize}
\tightlist
\item
  Try to distribute participant ages across the full range of ages if possible

  \begin{itemize}
  \tightlist
  \item
    Please report the exact age in days. You can use an online tool such as: \url{https://www.calculator.net/age-calculator.html}
  \end{itemize}
\item
  In situations where labs planned but were unable to collect the minimum contribution, laboratory members will still be eligible for authorship. Alternative contributions may be requested by the Leadership Team in these circumstances.
\item
  A lab's sample size contribution includes any infant who enters the laboratory, even if they are eventually excluded (e.g., due to fussiness, experimenter error, etc).
\end{itemize}

While we encourage as much participation as you can spare, it is crucial to the success of the project that you treat our study with the same care as you would any other study in your lab with respect to recruitment procedures, timing of data collection, and RAs/RA training. Please do not commit to providing data to ManyBabies if this level of care is not possible.

\hypertarget{eligibility-and-exclusions}{%
\subsection{Eligibility and Exclusions}\label{eligibility-and-exclusions}}

\textbf{The only requirements for inclusion in MB5 are that an infants' age falls within the 3- to 15-month age range and that the infant has no known issues that would directly impede their ability to process visual stimuli.} Infants who are hard of hearing, premature, bilingual, etc can be included in the lab's sample.

Please determine participants' eligibility on the phone prior to scheduling them in the ManyBabies study to avoid testing ineligible participants, but note that the inclusion criteria for MB5 are much more inclusive than typical in-lab studies (described above). Participants who come into the lab who are subsequently determined to be ineligible should be reported in the sample.

\hypertarget{first-sessionsecond-session-policy}{%
\subsection{First-session/second-session policy}\label{first-sessionsecond-session-policy}}

Some laboratories have the practice of testing babies in more than one study during the same visit. \textbf{\emph{`First session'}} refers to babies tested soon upon arrival to the lab, prior to participating in any other studies. \textbf{\emph{`Second session'}} refers to any testing that is done after the first session.
* \textbf{Please contribute `first session' babies when possible.} It is possible that second session babies will contribute worse/weaker/different data with respect to the larger goals of determining effect sizes.
* \textbf{Please label `second session' babies} appropriately if infants were run in a different study on the same visit prior to their participation in MB5. Please also document the nature of the study run prior to MB5 for all `second session' participants (e.g., ``7 minute study of object categorization using eye-tracking'').

\hypertarget{setting-up-the-experiment}{%
\chapter{Setting up the experiment}\label{setting-up-the-experiment}}

Cross-references make it easier for your readers to find and link to elements in your book.

\hypertarget{chapters-and-sub-chapters}{%
\section{Chapters and sub-chapters}\label{chapters-and-sub-chapters}}

There are two steps to cross-reference any heading:

\begin{enumerate}
\def\labelenumi{\arabic{enumi}.}
\tightlist
\item
  Label the heading: \texttt{\#\ Hello\ world\ \{\#nice-label\}}.

  \begin{itemize}
  \tightlist
  \item
    Leave the label off if you like the automated heading generated based on your heading title: for example, \texttt{\#\ Hello\ world} = \texttt{\#\ Hello\ world\ \{\#hello-world\}}.
  \item
    To label an un-numbered heading, use: \texttt{\#\ Hello\ world\ \{-\#nice-label\}} or \texttt{\{\#\ Hello\ world\ .unnumbered\}}.
  \end{itemize}
\item
  Next, reference the labeled heading anywhere in the text using \texttt{\textbackslash{}@ref(nice-label)}; for example, please see Chapter \ref{cross}.

  \begin{itemize}
  \tightlist
  \item
    If you prefer text as the link instead of a numbered reference use: \protect\hyperlink{cross}{any text you want can go here}.
  \end{itemize}
\end{enumerate}

\hypertarget{captioned-figures-and-tables}{%
\section{Captioned figures and tables}\label{captioned-figures-and-tables}}

Figures and tables \emph{with captions} can also be cross-referenced from elsewhere in your book using \texttt{\textbackslash{}@ref(fig:chunk-label)} and \texttt{\textbackslash{}@ref(tab:chunk-label)}, respectively.

See Figure \ref{fig:nice-fig}.

\begin{Shaded}
\begin{Highlighting}[]
\FunctionTok{par}\NormalTok{(}\AttributeTok{mar =} \FunctionTok{c}\NormalTok{(}\DecValTok{4}\NormalTok{, }\DecValTok{4}\NormalTok{, .}\DecValTok{1}\NormalTok{, .}\DecValTok{1}\NormalTok{))}
\FunctionTok{plot}\NormalTok{(pressure, }\AttributeTok{type =} \StringTok{\textquotesingle{}b\textquotesingle{}}\NormalTok{, }\AttributeTok{pch =} \DecValTok{19}\NormalTok{)}
\end{Highlighting}
\end{Shaded}

\begin{figure}

{\centering \includegraphics[width=0.8\linewidth]{_main_files/figure-latex/nice-fig-1} 

}

\caption{Here is a nice figure!}\label{fig:nice-fig}
\end{figure}

Don't miss Table \ref{tab:nice-tab}.

\begin{Shaded}
\begin{Highlighting}[]
\NormalTok{knitr}\SpecialCharTok{::}\FunctionTok{kable}\NormalTok{(}
  \FunctionTok{head}\NormalTok{(pressure, }\DecValTok{10}\NormalTok{), }\AttributeTok{caption =} \StringTok{\textquotesingle{}Here is a nice table!\textquotesingle{}}\NormalTok{,}
  \AttributeTok{booktabs =} \ConstantTok{TRUE}
\NormalTok{)}
\end{Highlighting}
\end{Shaded}

\begin{table}

\caption{\label{tab:nice-tab}Here is a nice table!}
\centering
\begin{tabular}[t]{rr}
\toprule
temperature & pressure\\
\midrule
0 & 0.0002\\
20 & 0.0012\\
40 & 0.0060\\
60 & 0.0300\\
80 & 0.0900\\
\addlinespace
100 & 0.2700\\
120 & 0.7500\\
140 & 1.8500\\
160 & 4.2000\\
180 & 8.8000\\
\bottomrule
\end{tabular}
\end{table}

\hypertarget{general-lab-practices}{%
\chapter{General Lab Practices}\label{general-lab-practices}}

{[}note: this could maybe be a more comprehensive section that includes info about offline coding{]}

You can add parts to organize one or more book chapters together. Parts can be inserted at the top of an .Rmd file, before the first-level chapter heading in that same file.

Add a numbered part: \texttt{\#\ (PART)\ Act\ one\ \{-\}} (followed by \texttt{\#\ A\ chapter})

Add an unnumbered part: \texttt{\#\ (PART\textbackslash{}*)\ Act\ one\ \{-\}} (followed by \texttt{\#\ A\ chapter})

Add an appendix as a special kind of un-numbered part: \texttt{\#\ (APPENDIX)\ Other\ stuff\ \{-\}} (followed by \texttt{\#\ A\ chapter}). Chapters in an appendix are prepended with letters instead of numbers.

\hypertarget{prior-to-data-collection-checklist}{%
\chapter{Prior to data collection checklist}\label{prior-to-data-collection-checklist}}

\hypertarget{footnotes}{%
\section{Footnotes}\label{footnotes}}

Footnotes are put inside the square brackets after a caret \texttt{\^{}{[}{]}}. Like this one \footnote{This is a footnote.}.

\hypertarget{citations}{%
\section{Citations}\label{citations}}

Reference items in your bibliography file(s) using \texttt{@key}.

For example, we are using the \textbf{bookdown} package \citep{R-bookdown} (check out the last code chunk in index.Rmd to see how this citation key was added) in this sample book, which was built on top of R Markdown and \textbf{knitr} \citep{xie2015} (this citation was added manually in an external file book.bib).
Note that the \texttt{.bib} files need to be listed in the index.Rmd with the YAML \texttt{bibliography} key.

The RStudio Visual Markdown Editor can also make it easier to insert citations: \url{https://rstudio.github.io/visual-markdown-editing/\#/citations}

\hypertarget{during-data-collection}{%
\chapter{During data collection}\label{during-data-collection}}

\hypertarget{equations}{%
\section{Equations}\label{equations}}

Here is an equation.

\begin{equation} 
  f\left(k\right) = \binom{n}{k} p^k\left(1-p\right)^{n-k}
  \label{eq:binom}
\end{equation}

You may refer to using \texttt{\textbackslash{}@ref(eq:binom)}, like see Equation \eqref{eq:binom}.

\hypertarget{theorems-and-proofs}{%
\section{Theorems and proofs}\label{theorems-and-proofs}}

Labeled theorems can be referenced in text using \texttt{\textbackslash{}@ref(thm:tri)}, for example, check out this smart theorem \ref{thm:tri}.

\begin{theorem}
\protect\hypertarget{thm:tri}{}\label{thm:tri}For a right triangle, if \(c\) denotes the \emph{length} of the hypotenuse
and \(a\) and \(b\) denote the lengths of the \textbf{other} two sides, we have
\[a^2 + b^2 = c^2\]
\end{theorem}

Read more here \url{https://bookdown.org/yihui/bookdown/markdown-extensions-by-bookdown.html}.

\hypertarget{callout-blocks}{%
\section{Callout blocks}\label{callout-blocks}}

The R Markdown Cookbook provides more help on how to use custom blocks to design your own callouts: \url{https://bookdown.org/yihui/rmarkdown-cookbook/custom-blocks.html}

\hypertarget{after-data-collection}{%
\chapter{After data collection}\label{after-data-collection}}

\hypertarget{publishing}{%
\section{Publishing}\label{publishing}}

HTML books can be published online, see: \url{https://bookdown.org/yihui/bookdown/publishing.html}

\hypertarget{pages}{%
\section{404 pages}\label{pages}}

By default, users will be directed to a 404 page if they try to access a webpage that cannot be found. If you'd like to customize your 404 page instead of using the default, you may add either a \texttt{\_404.Rmd} or \texttt{\_404.md} file to your project root and use code and/or Markdown syntax.

\hypertarget{metadata-for-sharing}{%
\section{Metadata for sharing}\label{metadata-for-sharing}}

Bookdown HTML books will provide HTML metadata for social sharing on platforms like Twitter, Facebook, and LinkedIn, using information you provide in the \texttt{index.Rmd} YAML. To setup, set the \texttt{url} for your book and the path to your \texttt{cover-image} file. Your book's \texttt{title} and \texttt{description} are also used.

This \texttt{gitbook} uses the same social sharing data across all chapters in your book- all links shared will look the same.

Specify your book's source repository on GitHub using the \texttt{edit} key under the configuration options in the \texttt{\_output.yml} file, which allows users to suggest an edit by linking to a chapter's source file.

Read more about the features of this output format here:

\url{https://pkgs.rstudio.com/bookdown/reference/gitbook.html}

Or use:

\begin{Shaded}
\begin{Highlighting}[]
\NormalTok{?bookdown}\SpecialCharTok{::}\NormalTok{gitbook}
\end{Highlighting}
\end{Shaded}

\hypertarget{additional-chapter}{%
\chapter{Additional Chapter}\label{additional-chapter}}

\hypertarget{heres-the-next-chapter}{%
\section{Here's the next chapter}\label{heres-the-next-chapter}}

Wow, really insightful!!

  \bibliography{refs.bib,packages.bib}

\end{document}
